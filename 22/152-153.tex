\documentclass[twoside, a4paper,11pt]{report}
\usepackage[T1,T2A]{fontenc}
\usepackage[utf8]{inputenc} 
\usepackage[english, russian]{babel}
\usepackage{amsmath}
\usepackage{fancyhdr}

\renewcommand{\headrulewidth}{0pt}
\renewcommand{\footrulewidth}{0pt}
\renewcommand{\headheight}{14pt}

\fancyhf{} 
\fancyhead[R]{77}
\fancyhead[L]{yz}
\fancyhead[C]{hello my dear friend}






\setlength{\parindent}{10mm}
\makeatletter
% Переопределение команды секции
\renewcommand{\section}{\@startsection{section}{1}%
{\parindent}{3.25ex plus 1ex minus .2ex}%
{1.5ex plus .2ex}{\bfseries\large}}

% Переопределение команды подсекции
\renewcommand{\subsection}{\@startsection{subsection}{2}%
{\parindent}{3.25ex plus 1ex minus .2ex}%
{1.5ex plus .2ex}{\bfseries}}
\makeatother

\begin{document}
\pagestyle{fancy} 


\noindent от непрерывной функции. Следовательно,
$$
\sum ^{i}_{n=2}=\dfrac {\left( 36\right) }{\left( 2-3\right) ^{\left( 2+6\right) }}
$$

    Рассмотрим задачу минимизации первого, главного
члена выражения (4). Для удобство решения уравнения 
Эйлера примем за независимую переменную функцию   

\subsection*{\S 10. Примеры оптимизации распределения узлов}
Рассмотрим пример решения уравнения (9.6)

\end{document}
