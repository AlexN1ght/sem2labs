\documentclass[twoside]{article}
\usepackage[14pt]{extsizes}
\usepackage[T1,T2A]{fontenc}
\usepackage[utf8]{inputenc} 
\usepackage[english, russian]{babel}
\usepackage[a4paper, left=25mm, right=35mm, top=20mm, bottom=30mm]{geometry}
\usepackage{lipsum}
\usepackage{amsmath}
\usepackage{fancyhdr}

\newcommand{\RNumb}[1]{\uppercase\expandafter{\romannumeral #1\relax}}

\renewcommand{\headrulewidth}{0pt}
\renewcommand{\footrulewidth}{0pt}
\setlength{\headheight}{14pt}
\fontdimen2\font=1.4ex
\linespread{0.8}
\sloppy

\setcounter{page}{152}

\fancyhf{} 
\fancyhead[LE,RO]{\thepage}
\fancyhead[LO]{\S 10}
\fancyhead[RE]{ГЛ. \RNumb{3}}
\fancyhead[CE]{ЧИСЛЕННОЕ ИНТЕГРИРОВАНИЕ}
\fancyhead[CO]{ПРИМЕРЫ ОПТИМИЗАЦИИ РАСПРЕДЕЛЕНИЯ УЗЛОВ}


\setlength{\parindent}{10mm}
\makeatletter
% Переопределение команды секции
\renewcommand{\section}{\@startsection{section}{1}%
{\parindent}{3.25ex plus 1ex minus .2ex}%
{1.5ex plus .2ex}{\bfseries\large}}

% Переопределение команды подсекции
\renewcommand{\subsection}{\@startsection{subsection}{2}%
{\parindent}{3.25ex plus 1ex minus .2ex}%
{1.5ex plus .2ex}{\bfseries}}
\makeatother

\begin{document}
\pagestyle{fancy} 


\noindent от непрерывной функции. Следовательно,
$$
\overline {r}=\dfrac {1}{N^{2}}\left( \int ^{1}_{0}\left( \varphi '\left( t\right) \right) ^{3}\dfrac {F\left( \varphi \left( t\right) \right) }{12}dt\right) +o\left( \dfrac {1}{N^{2}}\right).\eqno(4)\
$$

    Рассмотрим задачу минимизации первого, главного
члена выражения (4). Для удобство решения уравнения 
Эйлера примем за независимую переменную функцию $\varphi$.
Тогда коэффициент при $\dfrac {1}{N^{2}}$ в главном члене погрешности
запишется в виде 
$$
\int ^{1}_{0}\left( t'\left( \varphi \right) \right) ^{-2}\dfrac {F\left( \varphi \right) }{12}d\varphi .
$$
Уравнение  Эйлера для функции, минимизирующей функционал
$$
\int ^{1}_{0}G\left( \varphi ,t,t'\right) d\varphi ,
$$
имеет вид
$$
\dfrac {d}{d\varphi }\left( \dfrac {\partial G}{\partial t'}\right) -\dfrac {\partial G}{\partial t}=0.\eqno(5)
$$
В рассматриваемом  случае $\dfrac {\partial G}{\partial t}=0$, и
поэтому из (5) следует $\dfrac {\partial G}{\partial t'}=const.$ 
Подставляя конкретное значение функции $G,$ получим
$$
\left( t'\left( \varphi \right) \right) ^{-3}\dfrac {F\left( \varphi \right) }{6}=const
$$
или
$$
F\left( \varphi \right) \left( \varphi '\left( t\right) \right) ^{3}=C_{1} \eqno(6)
$$
Общее решения этого уравнения завистит от $C_{1}$ и еще и
некоторой постоянной $C_{2}$. Значения этих постоянныя можно
определить из граничных  условий $\varphi \left( 0\right) =\left( 0\right) ,\phi \left( 1\right) =1.$ 
Решение рассмотренной вариационной
постановки может практически испльзоваться различными способами.
Например, существуют программы, осуществляющие с=численное
интегрирование (6) на сетке с шагом, существенно юлдбшим $\dfrac {1}{N},$
и затем распределяющие узлы в соответствии с полученным решением.

Из соотношения (6) можно сделать тот же  вывод о равенстве оценок
погрешностей на элементарных отрезках интегрирования при оптимальном
распределении узлов. В самом деле, умножим (6) на $\dfrac {1}{12N^3},$ положим
$t=\dfrac {q}{N}$ и заменим $F\left( \varphi \left( \dfrac {q}{N}\right) \right)$
и $\dfrac {1}{N}\varphi '\left( \dfrac {q}{N}\right) ,$ соответственно, эквивалентными велечинами
\\ $\max _{\left[ a_{q-1},a_{q}\right] }F\left( x\right) ,a_{q}-a_{q-1}.$ В результате получим 
$$
\left( \max _{\left[ a_{q-1},a_{q}\right] }F\left( x\right) \right) \dfrac {\left( a_{q}-a_{q-1}\right) ^{3}}{12}\approx \dfrac {C_{1}}{12N^{3}}. \eqno(7)
$$

Еще один путь практического истользования управления (6) состоит в следующем.
Пусть требуется вычислить большую серию интегралов с одиноковым характерным 
поведением подынтыгральных функций. Выделим простейшую, модельную функцию, для которой задача
оптимизация узлов может быть решена в явном виде, и далее будем производить
интегрирование с распределением узлов, соответствующим этой функции. 
Если характер измерения функции рассматриваемой серии зависит от некоторого 
параметра, то этот параметр следует учесть при выборе модельной функции; естественно, 
что модельная функция не обязательно относится к рассматриваему классу. Чем большее 
число задач предъявляется для решения, тем более могут быть оправданы затраты, 
связанные с удвчным выбором и расмотрением модельной задачи.

\subsection*{\S 10. Примеры оптимизации распределения узлов}
Рассмотрим пример решения уравнения (9.6) для конкретных задач.

П\hspace{3pt}р\hspace{3pt}и\hspace{3pt}м\hspace{3pt}е\hspace{3pt}р 1. Пусть вычисляется серия интегралов
$$
I\left( b\right) =\int ^{1}_{0}f\left( b,x\right) dx,
$$
$b$ --- параметр серии, $f\left( b,x\right) =x^{b}g\left( b,x\right) ,$ --- $1 <b <2, g\left( b,x\right)$ ---
гладкая, $g\left( b,0\right) \neq 0.$ Если $b\neq 0,1,$ то вторая производная  $f_{xx}\left( b,x\right)$ не огранияена в окрестности точки 0; при выборе модельной
задачи следует учесть эту специфику поведения подынтегральной функции. В окрестности $x=0$ имеем
$$
f_{xx}\left( b,x\right) =b\left( b-1\right) x^{b-2}g\left( b,0\right) +O\left( x^{b-1}\right)
$$
Таким образом, в окрестности точки $x=0$ вторая производная $f_{xx}$ пропорциональна второй производной функции $x^b,$
поэтому функцию $x^b$ естественно расматривать в качестве модельной. Примим за $F(x)$ велечину 
$\left| b\left( b-1\right) \right| x^{b-2},$ тогда уравнение (9.6) запишется в виде
$$
\left| b\left( b-1\right) \right| \varphi ^{b-2}\left( \dfrac {d\varphi }{dt}\right) ^{3}=C_{1}.
$$

\end{document}
