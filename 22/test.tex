\documentclass{article}
\usepackage{amsmath,amsthm,amssymb}
\usepackage{mathtext}
\usepackage[T1,T2A]{fontenc}
\usepackage[utf8]{inputenc}
\usepackage[english,russian]{babel}
\begin{document}
	Вообще говоря, пакет babel автоматически выберет кодировку шрифта по умолчанию: для русского, болгарского и украинского языков это будет T2A. Однако для многоязыковых документов, в которых используются языки, основанные на кириллице и латинице, имеет смысл явно указать латинскую кодировку шрифтов. Пакет babel сам переключит нужную кодировку шрифта при смене языка в тексте документа.

	На современных операционных системах в качестве внутренней кодировки кириллических текстов лучше всего использовать Unicode (utf8 or utf8x), а не KOI8-RU (koi8-ru).
\end{document}
